% !TeX root = ../main.tex
% Add the above to each chapter to make compiling the PDF easier in some editors.

\chapter{Conclusion}\label{chapter:conclusion}

The Thesis contains the training of a Neural Network to predict the total farm power or expected farm power generated by a wind farm consisting of two turbines. The inputs to this the model are the relative position of the second wind turbine to the first as well as the wind condition parameters like wind direction at the given location. The resulting Neural Networks were embedded into a mixed integer linear programming problem using an approach for decomposing a Neural Network into mixed-integer linear constraints. Multiple optimization problems were solved using this approach, beginning with a deterministic problem by only considering a dominant wind direction to find maxima for the total farm power. The formulation was then expanded to stochastic optimizations with a scenario based approach, to maximize the expected power generation, subject to a discretized distribution of the wind direction. Here two variants were developed, one via a large general neural network with the wind condition parameters as additional variables and one via a reduced Neural Network, trained on the already evaluated expected farm power for each x/y position of the second turbine. 

While the deterministic approach could be quickly discarded as insufficient, both of the stochastic formulations yielded benefits and drawbacks. The first formulation, using a larger Neural Network independent of the wind condition distribution yielded a versatile variant, with a Neural Network that could be applied to any location and any wind condition distribution, while only being able to consider a low number of scenarios due to computational constraints. The second stochastic formulation evaded the computational limitations of solving a very large optimization problem by training its model directly on the farm power expectations, making both the Neural Network and the optimization model significantly more efficient, but losing generality due to the Neural Networks conditioning on the wind condition distribution.

Overall, the second stochastic formulation appears to be the only variant that sufficiently considers the randomness of the wind conditions like wind speed, while showing a path to scaling up computationally. In any case, results from both stochastic formulations were able to show how insufficient it would be to only consider a deterministic formulation.

With the second stochastic formulation having shown promising results in this thesis, the next steps would involve investigating how to generalize the formulation to a larger number of wind turbines. Here, a fundamental reformulation might be required, as the current approach using relative positions would cause the parameter space to grow exponentially. Nonetheless, this approach might be worthwhile to continue pursuing, as the underlying Constraint Learning Methodology has been shown to cope with the complexity of wind turbine wakes in this thesis.