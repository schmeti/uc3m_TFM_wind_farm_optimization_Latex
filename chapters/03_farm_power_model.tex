% !TeX root = ../main.tex
% Add the above to each chapter to make compiling the PDF easier in some editors.

\chapter{Farm Power Model}\label{chapter:introduction}

With the goal of training a model tailor made to the requirements of the optimization problem, the data used for training the model has to generated using a open source simulation method, allowing to set the parameter space of the model as required to the given optimization problem. After investigating the two python based wind farm simulation tools FLORIS and PyWake, FLORIS was chosen due to solid documentation, what appears to be very stable releases and broad functionalities regarding wake modelling. FLORIS is a wind farm simulation tool developed by the National Renewable Energy Laboratory (NREL).



Wind turbine power curve modelling under wake conditions using measurements from a spinner-mounted lidar:
https://www.sciencedirect.com/science/article/pii/S0306261924003684

Floris:
https://github.com/nrel/floris



Main Effects of Wake Turbolence: 
- Wind Speed (needs time to mix with outside airflow)
- Turbolence
\section{Data Source}

\section{Modelling}

\section{Validation}