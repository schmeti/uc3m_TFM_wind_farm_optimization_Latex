% !TeX root = ../main.tex
% Add the above to each chapter to make compiling the PDF easier in some editors.

\chapter{Conclusion}\label{chapter:conclusion}

The thesis contains the training of a Neural Network to predict the total farm power generated by a wind farm consisting of two turbines, with the inputs to the model as the relative position of the second wind turbine to the first as well as the wind direction. The resulting Neural Networks where embedded into a mixed integer linear programming problem using a novel approach for decomposing a Neural Network into linear constraints containing intenger variables and using a big M approach. Multiple optimization problems where solved using this aproach, beginning with a deterministic problem by only considering a dominant wind direction to find maxima for the total farm power and then proceeding to a stochastic scenario based optimization problem in which the expectd power generation is maximized, subject to a discreitzed distribution of the wind direction. 

The main challange of the work, beyond the correct implementation, is the rapid increase in the size of the optimization problem, for both an increase in Neural Network size and additional numbers of scenarios in the probabilistic case. These limitations prevented the Neural Network to be set up with completely satisfactory results and further constrained the possible number of scenarios that could be taken into account. 

Meanwhile, the used was in general terms successfull in introducing a highly nonlinear relationship represented by the Neural Network into a mixed integer linear programming problem. The challenges found in doing so however appear to limit the scalability beyond the 2 turbine setup as a further increase in Neural Network size would be required if a third turbine in the form of its relative position would again increase the parameter space significantly. 

Overall, the method used in this work appears was shown to be able cope with highly nonlinear relationships, but in its current form appeared to reach its limits in the problem at hand. 

Among others, three potential pathways could be chosen to proceed from this point on and build on the work done in this thesis: 

\begin{enumerate}
	\item The probabilistic component of the problem could be moved from outside the model as the currently used scenario based approach could be replaced by changing the output of the Neural network to be the expected farm power. This would require applying the wind distribution to the raw simulation data first and thus remove the models independence of the wind distribution at a given location, but instead prevent the model from having to be embedded multiple times for scenarios and reduce the size of the required Neural Network as size of the parameter 
	\item Use the scenario based approach and attempt to replace the complex Neural Network by a simpler model type like an analytical or functional approximation 
\end{enumerate}


In any case, maximizing the expected farm power across the given wind direction distribution and, once computational constraints are resolved, using multivariated distribution of the wind conditions by adding potentiall interdependent random variables like windspeed seem feasible and significantly superior compared to only relying on a dominant wind direction. 