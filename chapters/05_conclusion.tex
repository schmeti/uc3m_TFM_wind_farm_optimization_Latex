% !TeX root = ../main.tex
% Add the above to each chapter to make compiling the PDF easier in some editors.

\chapter{Comparison of Methods and Conclusion}\label{chapter:conclusion}

The Thesis contains the training of a Neural Network to predict the total farm power/expected farm power generated by a wind farm consisting of two turbines, with the inputs to the model as the relative position of the second wind turbine to the first as well as the wind condition parameters like wind direction. The resulting Neural Networks where embedded into a mixed integer linear programming problem using an approach for decomposing a Neural Network into lmixed-integer inear constraints. Multiple optimization problems where solved using this aproach, beginning with a deterministic problem by only considering a dominant wind direction to find maxima for the total farm power. The formulation was then expanded to stochastic optimizations with a scenario based approach, to maximize the expected power generation, subject to a discreitzed distribution of the wind direction. Here two variants where developed, one via a large general neural network with the wind condition parameters as additional variables and one via a reduced Neural Network, trained on the already evaluated expected farm power for each x/y position of the second turbine. 

While the deterministic approach could be quickly discarded as insufficient, both of the stochastic formulations yielded benefits and drawbacks. The first formulation, using a larger Neural Network indipendent of the wind condition distribution yielded a versatile variant, with a Neural Network that could be applied to any location and any wind condition distribution, while only being able to consider a low number of scenarios due to computational constraints. The second stochastic formulation evaded the computational limitations by training its model directly on the farm power expectations, making both the Neural Network and the optimization model significantly more efficient, but loosing generality due to the Neural Networks conditioning on the wind condistion distribution.

Overall, the second stochastic formulation appears to be the only variant that sufficiently considers the randomness of the wind conditions like wind speed, while showing a path to scaling up computationally. In any case, results from both stochastic formulations where able to show how insufficient it would be to only consider a deterministic formulation.

With especially the second stochastic formulation having shown to deliver good results in this thesis, the next steps would revolve around investigating how to generalize the formulation to a larger number of wind turbines. Here, a fundamental reformulation might be required as the current formulation using relative positions would quickly grow the parameter space exponentially. 

None the less it might be worthwile to pursue as the underlying Constraint Learning Methodology has been shown to be able to cope with the complexity of wind turbine wakes in this Thesis.